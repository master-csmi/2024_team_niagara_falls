\documentclass[10pt]{beamer}
\usetheme{Warsaw}
\usepackage[T1]{fontenc}
\usepackage[utf8]{inputenc}
\usepackage{chronosys}

\begin{document}
\begin{frame}
    \tableofcontents
\end{frame}

% Partie 1:


\section{Introduction and History}
\subsection{Introduction}
\begin{frame}
\frametitle{Introduction and History}
\framesubtitle{Introduction}
\begin{itemize}
    \item Linear sequential phases
    \item Used in engineering design
    \item In software development : 
    \begin{itemize}
        \item earliest SDLC approach
        \item the less iterative and flexible approaches
    \end{itemize}    
\end{itemize}
\end{frame}

\subsection{History}
\begin{frame}
\frametitle{Introduction and History}
\framesubtitle{History}
\startchronology
[startyear=1950,stopyear=1990]
\chronoevent[textwidth=3cm]{1956}{\footnotesize First presentation}
\chronoevent[textwidth=3cm]{1970}{\footnotesize First citation in an article}
\chronoevent[textwidth=3cm]{1985}{\footnotesize United States Department of Defense captured this approach}
\stopchronology
    
\end{frame}


% Partie 2:

\begin{frame}[plain]
\setbeamertemplate{blocks}[rounded][shadow=false]
\frametitle{Requirement}

\begin{block}{system requirement}
The first part of this method is about the plan of the product, the price, the deadline and everything out of the product creation
\end{block}

\begin{block}{software requirement}
in this phase we need to know what will the product and document everything about it, functionality of the product, interface , support of the product  
\end{block}
\end{frame}
\begin{frame}[plain]
\setbeamertemplate{blocks}[rounded][shadow=false]
\frametitle{Analysis and Design}

\begin{block}{Analysis}
In this phase we need to understand the project and structure it to generate a model that will be used in the implementation. In this phase we also need to know the technical resources that will be used (for example the server for an application )
\end{block}
\begin{block}{Design}
The design is the phase we choose the details about implementation such as the language used, the class and libraries used for next phase

\end{block}
\end{frame}


% Partie 3:
\begin{frame}[plain]
\setbeamertemplate{blocks}[rounded][shadow=false]
\frametitle{Coding}
\begin{block}{Coding}
At this stage we start implementing the project,using the model and logic found during the last phase.The project will most likely be coded in smaller components before being put together.
\end{block}

\begin{block}{Testing}
After coding we need to test our product to see if it works well,do some quality insurance and debug.
\end{block}
\end{frame}

\begin{frame}[plain]
\setbeamertemplate{blocks}[rounded][shadow=false]
\frametitle{Last operation}
\begin{block}{Deployment}
The product is judged finished and deployed into action.
\end{block}
\begin{block}{Maintenance}
Correction of bug and performance maintenance to improve or fix the final product. That can lead to a series of patches. 
\end{block}
    

\end{frame}


% Partie 4:

\end{document}
