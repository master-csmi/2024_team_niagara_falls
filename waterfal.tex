\documentclass[10pt]{beamer}
\usetheme{Warsaw}
\usepackage[T1]{fontenc}
\usepackage[utf8]{inputenc}
\usepackage{chronosys}

\title{Waterfall}
\author{Team Niagara_falls}
\date{}

\begin{document}
\frame{\titlepage}
\begin{frame}
    \tableofcontents
\end{frame}

% Partie 1:


\section{Introduction and History}
\subsection{Introduction}
\begin{frame}
\frametitle{Introduction and History}
\framesubtitle{Introduction}
\begin{itemize}
    \item Linear sequential phases
    \item Used in engineering design
    \item In software development : 
    \begin{itemize}
        \item earliest SDLC approach
        \item the less iterative and flexible approaches
    \end{itemize}    
\end{itemize}
\end{frame}

\subsection{History}
\begin{frame}
\frametitle{Introduction and History}
\framesubtitle{History}
\startchronology
[startyear=1950,stopyear=1990]
\chronoevent[textwidth=3cm]{1956}{\footnotesize First presentation}
\chronoevent[textwidth=3cm]{1970}{\footnotesize First citation in an article}
\chronoevent[textwidth=3cm]{1985}{\footnotesize United States Department of Defense captured this approach}
\stopchronology
    
\end{frame}


% Partie 2:

\begin{frame}[plain]
\setbeamertemplate{blocks}[rounded][shadow=false]
\frametitle{Requirement}

\begin{block}{system requirement}
The first part of this method is about the plan of the product, the price, the deadline and everything out of the product creation
\end{block}

\begin{block}{software requirement}
in this phase we need to know what will the product and document everything about it, functionality of the product, interface , support of the product  
\end{block}
\end{frame}
\begin{frame}[plain]
\setbeamertemplate{blocks}[rounded][shadow=false]
\frametitle{Analysis and Design}

\begin{block}{Analysis}
In this phase we need to understand the project and structure it to generate a model that will be used in the implementation. In this phase we also need to know the technical resources that will be used (for example the server for an application )
\end{block}
\begin{block}{Design}
The design is the phase we choose the details about implementation such as the language used, the class and libraries used for next phase

\end{block}
\end{frame}


% Partie 3:
\begin{frame}[plain]
\setbeamertemplate{blocks}[rounded][shadow=false]
\frametitle{Coding}
\begin{block}{Coding}
At this stage we start implementing the project,using the model and logic found during the last phase.The project will most likely be coded in smaller components before being put together.
\end{block}

\begin{block}{Testing}
After coding we need to test our product to see if it works well,do some quality insurance and debug.
\end{block}
\end{frame}

\begin{frame}[plain]
\setbeamertemplate{blocks}[rounded][shadow=false]
\frametitle{Last operation}
\begin{block}{Deployment}
The product is judged finished and deployed into action.
\end{block}
\begin{block}{Maintenance}
Correction of bug and performance maintenance to improve or fix the final product. That can lead to a series of patches. 
\end{block}
    

\end{frame}


% Partie 4:

\section{Advantages and Criticisms}
\subsection{Pros:}

\begin{frame}[plain]
\setbeamertemplate{blocks}[rounded][shadow=false]
\frametitle{Advantages}

\begin{block}{Simple and easy to understand}
Its linear and sequential nature makes it easy to comprehend, especially for stakeholders who are not familiar with software development processes.
\end{block}

\begin{block}{Clear milestones and deliverables}
Each phase has well-defined deliverables and milestones, making it easier to track progress and manage expectations.
\end{block}

\begin{block}{Early detection of issues}
Because requirements are established upfront, any potential issues can be identified early in the process, reducing the likelihood of major changes later on.
\end{block}

\begin{block}{Structured approach}
The rigid structure ensures that each phase is completed before moving on to the next, which can provide a sense of security and stability.
\end{block}

\end{frame}

\subsection{Cons:}

\begin{frame}[plain]
\setbeamertemplate{blocks}[rounded][shadow=false]
\frametitle{Criticisms}

\begin{block}{Limited flexibility}
The linear nature of the Waterfall model makes it difficult to accommodate changes once a phase is completed.
\end{block}

\begin{block}{Late testing}
Testing occurs towards the end of the development process, which means that defects may not be discovered until late stages, leading to higher costs and risks.
\end{block}

\begin{block}{Client involvement limited to early stages}
This involvement typically occurs primarily in the requirements phase, which can lead to misunderstandings or mismatches between client expectations and the final product.
\end{block}

\end{frame}




\end{document}
